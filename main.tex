\documentclass[11pt,a4paper]{article}

\usepackage{jheppub}
\usepackage{fontspec}
\usepackage{dsfont}
\usepackage{nicefrac}
\usepackage{bm}
\usepackage{mathsettings}
\usepackage{xparse}
\usepackage{mathrsfs}
\usepackage[none]{hyphenat}
\usepackage{microtype}
\usepackage{tabularx}
\usepackage{makecell}
\usepackage{empheq}
\usepackage{pifont}
\usepackage{hhline}
\usepackage{tikz}
\usepackage[
	backend=biber,
	style=numeric,
	sorting=none,
	language=australian
]{biblatex}

\addbibresource{bibliography.bib}

\graphicspath{{./figures/}}

\usetikzlibrary{ext.paths.ortho,  % -|- and |-| path operations
                quotes,
                tikzmark,
                }
\tikzset{
 is/.style = {inner ysep=2pt},
lbl/.style = {anchor=#1, inner sep=1pt, align=center,
              font=\footnotesize\linespread{0.84}\selectfont},
lbl/.default=south
        }


\newcommand{\bfnt}[1]{{\bfseries #1}}
\newcommand{\ifnt}[1]{{\itshape #1}}

% https://tex.stackexchange.com/a/531829/200495
\makeatletter
\gdef\@fpheader{}
\makeatother

% Package siunitx has deprecated Angstrom.
% https://tex.stackexchange.com/q/610292/200495
\DeclareSIUnit\angstrom{\text{\AA}}

% https://tex.stackexchange.com/a/65934/200495
\newcommand\Tstrut{\rule{0pt}{2.6ex}}       % top strut
\newcommand\Bstrut{\rule[-1.1ex]{0pt}{0pt}} % bottom strut

% Courtsey: https://tex.stackexchange.com/a/326380/200495
% Syntax: \colorboxed[<color model>]{<color specification>}{<math formula>}
\makeatletter
\newcommand*{\colorboxed}{}
\def\colorboxed#1#{%
  \@colorboxed{#1}%
}
\newcommand*{\@colorboxed}[3]{%
  % #1: optional argument for color model
  % #2: color specification
  % #3: formula
  \begingroup
    \colorlet{cb@saved}{.}%
    \color#1{#2}%
    \boxed{%
      \color{cb@saved}%
      #3%
    }%
  \endgroup
}
\makeatother

% Provides \Acolorboxed[color]{equation with align character (&)}. 
% Colored variant of \Acolorboxed{} from mathtools package.
% Courtsey:https://tex.stackexchange.com/a/610299/200495
\makeatletter
\newcommand*\Acolorboxed[2][red]{%
   \let\bgroup{\romannumeral-`}%
   \@Acolorboxed{#1}#2&&\ENDDNE
}
\def\@Acolorboxed#1#2&#3&#4\ENDDNE{%
  \ifnum0=`{}\fi
  \setbox\z@\hbox{$\displaystyle#2{}\m@th$\kern\fboxsep \kern\fboxrule}%
  \edef\@tempa{\kern\wd\z@ & \kern-\the\wd\z@ \fboxsep\the\fboxsep \fboxrule\the\fboxrule}%
  \@tempa
  \fcolorbox{#1}{white}{\m@th$\displaystyle#2#3$}%
} 
\makeatother


\begin{document}

	\title{A Brief Primer on X-ray Diffraction Crystallography}
	\author{Wrichik Basu}
	\affiliation{Semester VI, B.Sc. Physics (Hons.)\\Dept. of Physics, Scottish Church College\\University of Calcutta, India.}
	\date{\today}
	
	\maketitle
	\flushbottom
	
	\section{Introduction}

X-ray crystallography is the experimental science of determining the structure of crystals using X-rays. The discovery of diffraction of X-rays by crystals by Max von Laue, and subsequent work by the Braggs, Ewald and Scherrer, opened a vast realm in science. A large number of substances in the nature, including biological molecules, can crystallize, and hence can be studied using this technique. Not only can we determine the crystal structure of the sample, we can also find the structure of the material under study. In this writing, we start by first talking about the generation and sources of X-rays. Next, we briefly dive into the theory of X-ray diffraction before entering the experimental realm. Thereafter, we discuss which type of X-rays are used and the types of XRD experiments. Then comes the two major types of XRD experiments. We conclude with the application of XRD in nanophysics. We hope that you will enjoy this journey. Without further ado, let us begin.
	
	\section{Spectroscopy vs. Crystallography}
	
\textbf{Spectroscopy} is the study of absorption and/or emission of electromagnetic radiation in which the incident radiation interacts with the molecule and produces characteristic signals. From spectroscopy, we can get the following data about a molecule:%
%			
\begin{itemize}%
%		
    \item Bond lengths and bond angles of simple (diatomic, triatomic) molecules.
    
    \item Combination of Infrared and NMR (${}^1 \mathrm{H}$ and ${}^{13} \mathrm{C}$) spectroscopy provides information about functional groups, bond connectivity and stereochemistry (absolute configuration) of simple molecules.
    
    \end{itemize}
    
Spectroscopy, however, cannot provide information about bond lengths, bond angles, torsion angles, etc.\ of complicated molecules.

\textbf{Crystallography}, on the other hand, deals in the interaction of EM radiation of very small wavelength ($\lambda \sim \SI{1}{\angstrom}$) with solid materials (crystalline or amorphous) through scattering. The main difference with spectroscopy is that in crystallography, there is no absorption or emission of radiation; only scattering of radiation. Crystallography allows us to study%
%			
\begin{itemize}%
%			
    \item Accurate bond length, bond angles, torsion angles, precise absolute configuration, etc.\ of all types of crystalline substances.
    
    \item 3D structure of crystalline substances can lead to intermolecular or inter-ionic interactions. Structure property relationship can be obtained using data from crystallography.
    
\end{itemize}
	
	\section{Why X-rays?}

		Simply because the wavelength of X-rays is of the same order as the inter-atomic distance in a crystal lattice.
		
	\section{Sources of X-rays}

	For X-ray diffraction experiments, monochromatic radiation from various metal-based sources (eg. Cu, Ag, Mo) is used in the lab. There are also synchrotron-based sources for X-rays.%
%			
	\begin{itemize}%
%			
	    \item \bfnt{Sealed-tube X-ray sources}%
%			    	
	    	\begin{itemize}[label={$\hookrightarrow$}]%
%			    	
	    	    \item \ul{Fine focus sources}: Operate at around $50~\si{kV}$ and $40~\si{mA}$ ($\sim \SI{2}{kW}$). Can give a photon flux of around $\SI{1e7}{photons/s~mm^2}.$
	    	    
	    	    \item \ul{Microfocus sources}: Highly focussed beam. Operates at $40-50~\si{kV}$ and $2-15~\si{mA}$  ($<\SI{1}{kW}$). Can provide around $\SI{1e8}{photons/s~mm^2}.$
	    	    
	    	\end{itemize}
	    	
	    \item \bfnt{Rotating anode based source}: The anode is rotated at a very high speed of around $10,000~\si{rpm}.$ Cu, Mo or dual anode is used. Generally microfocus-based system. Operates at $\SI{60}{kV}$ and $\SI{100}{mA}$ or higher ($> \SI{6}{kW}$). Can produce a flux $\sim \num{1e10}-\num{e11}~\si{photons/s~mm^2}.$
	    
	    \item \bfnt{Metal Jet source}: Liquid Gallium is used, giving $\lambda = \SI{1.340}{\angstrom}.$ High intensity at much low power. Can generate $\sim \num{1e11}-\num{e12}~\si{photons/s~mm^2}.$
	    
	    \item \bfnt{Synchrotron Sources}: Synchrotron radiation is an EM radiation emitted when charged particles are subjected to an acceleration perpendicular to their velocity. In synchrotron, this radiation is generated in the dipole bending magnets, undulators and wigglers. This radiation has a characteristic polarization and the wavelengths generated can span the entire EM spectrum. X-rays from synchrotron sources are produce high quality data in powder X-ray diffraction experiments.
	    
	\end{itemize}
	
	\section{Generating X-rays}

\begin{figure}
	\centering
	\includegraphics[scale=0.14]{xray_tube.png}
	\caption{\label{fig:xray_tube}Schematic of a X-ray tube.}
\end{figure}

	Figure~\ref{fig:xray_tube} shows the schematic of a X-ray tube. The potential difference between the anode to cathode is around $20-60~\si{kV},$ while the Tungsten filament is supplied a current of $\sim 2-50~\si{mA}.$ The target is composed of the material from which we want to generate the X-rays (generally Cu, Mo or Ag). The Beryllium window provides a transparent region for the generated X-rays to pass through.
	
	The tube is not allowed to cool down between experiments. In the stand-by state, the filament current is reduced to around $\SI{5}{mA}$ and the anode potential is also reduced to $\SI{20}{kV}.$ When data collection is started, the anode voltage is increased to $>\SI{50}{kV}$ and the current in the Tungsten filament is increased to $\SI{40}{mA}.$ In this state, the heated Tungsten filament generates electrons, which are then accelerated and finally hit the target at the anode.

\begin{figure}[h]
	\centering
	\includegraphics[scale=0.1]{characteristic_xray_transitions.png}
	\caption{\label{fig:xray_transitions}Transitions giving rise to characteristic X-rays.}
\end{figure}
	
	These electrons are able to knock out electrons from the K-shell of an atom in the target. Once an electron is removed from the K-shell, electrons from higher energy levels release energy and come to the K-shell. This release energy is the characteristic X-rays of the material. There are three possible transitions which give rise to X-rays of three different wavelengths. These are shown in figure~\ref{fig:xray_transitions}.
	
\begin{figure}[h]
	\centering
	\includegraphics[scale=0.5]{xray_peaks.png}
	\caption{\label{fig:xray_spctra_Cu_Mo}X-ray spectra of $\mathrm{Mo}$ at $\SI{35}{kV}$. The $K_\alpha$ is shown expanded on the right. Picture courtesy:~\cite{Cullity2014}.}
\end{figure}
	
	The X-ray spectra of Mo is shown in figure~\ref{fig:xray_spctra_Cu_Mo}. The $K_\alpha$ peaks are very close to each other. The wavelengths can be found in table~\ref{tab:wavelengths}. The background radiation or white X-rays is attributed to Bremstrahlung, which is the radiation emitted by electrons when they are decelerated in the X-ray tube.
	
\begin{table}
	\centering
	\caption{\label{tab:wavelengths}Wavelengths of characteristic radiations of Cu and Mo.}
	\begin{tabular}{|c|C|C|C|C|C|c|}
	
		\hline
		
		Source & K_{\alpha_1} (\si{\angstrom}) & K_{\alpha_2} (\si{\angstrom}) & \multicolumn{1}{c|}{\makecell{$K_\alpha$ (average)\\($\si{\angstrom}$)}} & K_\beta (\si{\angstrom}) & Z & $\beta$ filter\\
		
		\hhline{|=|=|=|=|=|=|=|}
		
		Cu & 1.5405 & 1.5433 & 1.5418 & 1.3922 & 29 & Ni (Z = 28) \\
		
		\hline
		
		Mo & 0.7093 & 0.7136 & 0.7107 & 0.6393 & 42 & Nb (Z = 41)\\
		
		\hline
	
	\end{tabular}
\end{table}
	
	We use monochromatic radiation for X-ray diffraction experiments. For this, we have to filter out the unwanted $K_\beta$ and $K_{\alpha2}$ radiations. To filter the $K_\beta$ spectra, we use a $\beta$-filter. The  $\beta$-filters used for Cu and Mo are listed in table~\ref{tab:wavelengths}.
	
	To separate $K_{\alpha1}$ from $K_{\alpha2}$, we use a crystal monochromator. For this, a $\mathrm{Ge}$ crystal is cut along the $(111)$ plane, and the beam with mixed radiation is shined on this plane. Since the two radiations have different wavelengths, they will diffract at two different angles. We can eliminate the unwanted $K_{\alpha2}$ in this way. However, note that \ifnt{the intensity of the reflected beam will fall severely} due to loss of intensity upon reflection.
	
	\section{The physics behind X-ray diffraction}

Max von Laue discovered the phenomena of diffraction of X-rays by crystals, for which he was awarded the Nobel Prize in Physics in 1914. Laue viewed the three-dimensional periodic arrangement of atoms in a crystal as a 3D diffraction grating, and thereby derived a condition which needs to be satisfied for the diffraction of X-rays to take place.

Sir William Henry Bragg and William Lawrence Bragg furthered the physics of X-ray diffraction. Lawrence Bragg, in contrast to Laue's work, viewed the layers (planes) of atoms in the crystal lattice as reflecting surfaces, and X-ray beams reflecting off consecutive planes would interfere constructively if a certain condition was satisfied. This theory is not true in the physical sense -- planes of atoms do not reflect X-rays as such -- but it is correct in the geometrical sense, and we can arrive at Bragg's Law if we start from Laue's condition. The Braggs were jointly awarded the Nobel prize in Physics in 1915 for their work in X-ray crystallography.

\subsection{Bragg's Analysis}
	
	\begin{figure}
	\centering
	\includegraphics[scale=0.15]{bragg_law.png}
	\caption{\label{fig:bragg_law}X-ray beam is incident on a set of planes with Miller indices $(hk\ell),$ as shown. We draw perpendiculars $\mathrm{AB}$ and $\mathrm{AC}$. $d_{hk\ell}$ is the perpendicular distance between two planes with the same Miller indices $(hk\ell).$}
	\end{figure}

	Derivation of Bragg's Law is straightforward. Referring to figure~\ref{fig:bragg_law}, the path difference,%
%	
	\begin{align}
	\mathrm{P.D.} &= \mathrm{BD} + \mathrm{DC} \nonumber\\
				&= 2 d_{hk\ell} \sin \theta.
	\end{align}
	
	For constructive interference,%
%	
	\begin{align}
	&\phantom{\implies} \mathrm{P.D.} = n \lambda,\quad n \in \mathbb{Z} \nonumber\\
	&\implies \boxed{2 d_{hk\ell} \sin \theta = n \lambda}
	\end{align}
%	
	which is Bragg's Law for X-ray diffraction. $n$ is the order of diffraction or reflection.
	
%***************************************************************************************************
	
\subsection{Laue's analysis}

	\begin{figure}
	\centering
	\includegraphics[scale=0.17]{laue_analysis.png}
	\caption{\label{fig:laue_analysis}Diffraction from a lattice row along the $x$ axis. The incident and diffracted beams make angles $\alpha_0$ and $\alpha_n$ respectively with the lattice row. $\vec{a}$ is the lattice translation vector.}
	\end{figure}
	
	Consider a simple crystal with one atom as the basis, as shown in figure~\ref{fig:laue_analysis}. The atoms are regarded as the scattering centres from which the diffraction takes place. $\hat{S_0}$ and $\hat{S}$ are two unit vectors in the direction of the incident and diffracted beam, respectively. The lattice spacing is $a$.
	
	The path difference,%
%		
		\begin{align}
		\mathrm{P.D.} &= \mathrm{AB - CD}, \nonumber\\
					  &= a \qty( \cos \alpha_n - \cos \alpha_0 ) \\
					  &= \va{a} \cdot \qty( \vu{S} - \vu{S_0} ),
		\end{align}%
%		
	has to be an integer multiple of wavelength for constructive interference.%
%		
		\begin{equation}
		\therefore \boxed{a \qty( \cos \alpha_n - \cos \alpha_0 ) = \va{a} \cdot \qty( \vu{S} - \vu{S_0} ) = n_x \lambda,} \quad n_x \in \mathbb{Z}. \label{eq:1st_laue_eqn}
		\end{equation}
		
	Eqn.~\eqref{eq:1st_laue_eqn} is the \textbf{first Laue equation}.
	
	This path difference is still valid if the diffracted beam, instead of being below the row of atoms, is above it, or is out of the plane of the paper itself. Therefore, all diffracted beams with the same path difference occur at the same angle to the atom row. Thus, all diffracted beams lie on the surface of a cone, known as the \bfnt{Laue cone}, centred on the atom row with semi-apex angle $\alpha_n.$ This is illustrated in figure~\ref{fig:laue_cones}.
	
	\begin{figure}
	\centering
	\includegraphics[scale=0.17]{laue_cones.png}
	\caption{\label{fig:laue_cones}Three Laue cones representing the direction of diffracted beam from a lattice row along the $x$-axis, with $0$ ($n_x = 0$), $\lambda$ ($n_x = 1$) and $2\lambda$ ($n_x = 2$) path differences. Similar cones also lie to the left of the 0th order cone for $n < 0.$}
	\end{figure}
	
	Proceeding similarly, we can derive two more equations for diffraction from atom rows along the $y$ and $z$ directions, thereby arriving at the second and third Laue equations:%
%		
	\begin{align}
	\boxed{b \qty( \cos \beta_n - \cos \beta_0 ) = \va{b} \cdot \qty( \vu{S} - \vu{S_0} ) = n_y \lambda,} \quad n_y \in \mathbb{Z}; \label{eq:2nd_laue_eqn} \\[1.5em]
	\boxed{c \qty( \cos \gamma_n - \cos \gamma_0 ) = \va{a} \cdot \qty( \vu{S} - \vu{S_0} ) = n_z \lambda,} \quad n_z \in \mathbb{Z}. \label{eq:3rd_laue_eqn}
	\end{align}
	
	For constructive interference to simultaneously occur from all the three atom rows in the three directions, all three Laue equations must be satisfied simultaneously. This is geometrically equivalent to three Laue cones intersecting at some points. Diffraction will only occur along those directions in which three Laue cones intersect.
	
%***********************************************************************************************


	
	\section{The Choice of radiation}
	
	week 4, lec - 21 + week 5, lec - 22.
	
The maximum value of $\sin \theta = 1$ $\implies \theta_\mathrm{max} = \SI{90}{\degree}.$

\begin{align}
&\therefore \lambda = 2 \d_{hk\ell}^\mathrm{(min)} ~ \mathrm{for} ~ \theta = \pi/2 \nonumber \\[0.8em]
&\implies \d_{hk\ell}^\mathrm{(min)} = \dfrac{\lambda}{2} = \begin{dcases*}
\dfrac{1.54}{2} = \SI{0.77}{\angstrom} & for $\mathrm{Cu}~K_\alpha$ radiation, \\[0.8em]
\dfrac{0.71}{2} = \SI{0.35}{\angstrom} & for $\mathrm{Mo}~K_\alpha$ radiation.
\end{dcases*}
\end{align}

Thus, we see that we get a higher resolution when using $\mathrm{Mo}~K_\alpha$ radiation compared to $\mathrm{Cu}~K_\alpha$ radiation.

According to the standards set by IUCr, for $\mathrm{Mo}~K_\alpha$ radiation, data has to be recorded \ifnt{at least} upto $\SI{50}{\degree}$ in $2\theta$ for an acceptable crystal structure solution.

Substituting $\theta = \SI{25}{\degree}$ in Bragg's Law, for $\mathrm{Mo}~K_\alpha$, we get%
%
\begin{align}
&\phantom{\implies} \SI{0.71}{\angstrom} = 2 d_{hk\ell} \sin \SI{25}{\degree} \nonumber \\
&\implies d_{hk\ell} = \SI{1.183}{\angstrom}.
\end{align}

If we want to achieve this same resolution with $\mathrm{Cu}~K_\alpha,$%
%
\begin{align}
&\phantom{\implies} \SI{1.54}{\angstrom} = 2 \cross \SI{1.2}{\angstrom} \sin \theta \nonumber \\
&\implies \theta = \SI{42}{\degree} \nonumber \\
&\implies 2\theta = \SI{84}{\degree}.
\end{align}

Therefore, we have to record upto a higher angle to achieve the same resolution with $\mathrm{Cu}~K_\alpha.$

\begin{figure}[h!]
	\centering
	\includegraphics[scale=0.8]{pxrd_peaks.png}
	\caption{\label{fig:pxrd_peaks}Suppose we have recorded PXRD data using $\mathrm{Cu}~K_\alpha$ radiation from $\SI{3}{\degree}$ to $\SI{80}{\degree}$ in $2\theta$. The first peak appears at, say, $\SI{5}{\degree}$ and the last peak at $\SI{70}{\degree}.$ Image taken from \cite[lecture 22]{Chowdhury2022}.}
\end{figure}

Suppose we have a PXRD data as in figure~\ref{fig:pxrd_peaks}. Using Bragg's Law,%
%
\begin{align}
&\phantom{\implies} \dfrac{\lambda_\mathrm{Cu}}{\lambda_\mathrm{Mo}} = \dfrac{2 d_{hk\ell} \sin \theta_\mathrm{Cu}}{2 d_{hk\ell} \sin \theta_\mathrm{Mo}} \nonumber \\[0.8em]
&\implies \dfrac{\sin \theta_\mathrm{Cu}}{\sin \theta_\mathrm{Mo}} = \dfrac{1.54}{0.71}. \label{eq:sin_theta_Cu_Mo}
\end{align}

The first peak is at $\approx \SI{5}{\degree}$ in $2\theta_\mathrm{Cu}$ for $\mathrm{Cu}~K_\alpha$. Using eqn.~\eqref{eq:sin_theta_Cu_Mo}, we find that for $\mathrm{Mo}~K_\alpha,$ it will be at $2\theta_\mathrm{Mo} \approx \SI{2.3}{\degree}.$ Similarly, the last peak, which occurs at $2\theta_\mathrm{Cu} \approx \SI{70}{\degree},$ would have appeared at $2\theta_\mathrm{Mo} \approx \SI{30.66}{\degree}.$

Thus, when the data is recorded using $\mathrm{Cu}~K_\alpha$, the peaks appear in the range $\SI{5}{\degree}$ to $\SI{70}{\degree}$ in $2\theta,$ while for $\mathrm{Mo}~K_\alpha,$ they would have been squeezed between $\SI{2.3}{\degree}$ to $\SI{30.66}{\degree}$ in $2\theta.$ This will reduce the resolution of the peaks, and they may overlap or merge with each other, whereby we will lose information.

For PXRD, synchrotron source gives the best resolution. In the absence of that, we use $\mathrm{Cu}~K_\alpha.$ In SCXRD, where spots appear and they are well resolved, we can use $\mathrm{Mo}~K_\alpha$ radiation for higher resolution.

Note that monochromatic radiation is preferred because if we use polychromatic radiation, for a particular set of planes, there will be peaks for each wavelength in the X-ray diffraction spectrum, which are difficult to separate. In PXRD, however, often, we do not use a monochromator because it reduces the intensity drastically. In that situation, we have to identify the $K_{\alpha_1}$ and the $K_{\alpha_2}$ peaks in the spectra and separate them.
	
	\section{Type of X-ray diffraction experiments}

X-ray diffraction experiments for crystallography can be divided into two categories:%
%	
	\begin{itemize}%
%	
	    \item \bfnt{Powder XRD (PXRD)}:%
%	    	
	    	\begin{itemize}[label={$\rightarrowtail$}]%
%	    	
	    	    \item We get information about bulk properties like phase, purity, particle size, polymorph detection, etc.
	    	    
	    	    \item A bulk sample composed of micron or sub-micron sized particles is used.
	    	    
	    	    \item Small angle PXRD: $0-3~\si{\degree}$ in $2\theta.$ Wide angle PXRD: $3-80~\si{\degree}$ in $2\theta.$
	    	    
	    	    \item Generally, $\mathrm{Cu}~K_\alpha$ radiation is used.
	    	    
	    	\end{itemize}
	    	
	    \item \bfnt{Single Crystal XRD (SCXRD)}:%
%	    	
	    	\begin{itemize}[label={$\rightarrowtail$}]%
%	    	
	    	    \item Complete information about the crystal structure can be obtained.
	    	    
	    	    \item Slightly bigger crystals of size $5-50~\si{\micro\metre}$ have to be used.
	    	    
	    	    \item Generally $\mathrm{Mo}~K_\alpha$ radiation is used.
	    	    
	    	    \item $2-50~\si{\degree}$ in $2\theta$ sufficient for routine structure analysis. High resolution SCXRD requires $2\theta \in \qty[ \SI{2}{\degree}, \SI{120}{\degree} ].$
	    	    
	    	\end{itemize}
	    
	\end{itemize}
	
	\section{Single Crystal X-ray Diffraction}
	
		\subsection{Selection of crystals}

The criteria for selection of crystals for SCXRD is as follows:%
%			
	\begin{itemize}%
%			
	    \item \bfnt{Uniform internal structure}: The crystal should not have more than one domain of array of unit cells, should not be composed of a number of micron or sub-micron sized particles, and should not have crack or distortion of any means. However, \ifnt{the crystal need not have well defined faces}.
	    
	    \item \bfnt{Suitable size and shape}: For SCXRD experiments, chosen crystals should have dimensions in the range $50-500~\si{\micro\metre}.$ The size of the crystal must be smaller than the spot size of the X-ray beam.
	    
	    \begin{figure}
	    	\centering
	    	\includegraphics[scale=0.5]{imperfect_crystal.png}
	    	\caption{\label{fig:imperfect_crystal}The crystal on the left is a perfect crystal. The one on the right is a mosaic crystal with a deviation of $\sim 0.1-0.2 \si{\degree}.$ Image courtesy:~\cite{Chowdhury2022}.}
	    \end{figure}
	    
	    \item \bfnt{Imperfect crystals are better.} In perfect crystals, lattice planes traverse in the whole crystal without any deviation, resulting into extinction. Majority of crystals are imperfect, and they diffract better than perfect crystals. Imperfectness may be deliberately introduced into a crystal by a shock.
	    
	    \begin{figure}
	    	\centering
	    	\includegraphics[scale=0.4]{crystal_optical_microscope.png}
	    	\caption{\label{fig:crystal_optical_micro}Crystals when viewed through an optical microscope in polarized light. The set of crystals on the left are good crystals because they are either uniformly dark or uniformly bright when the analyzer is rotated by $\pi/2.$ The crystal on the right shows different colours when the analyzer is rotated, implying that it has more than one domain and cannot be studied under SCXRD. Image courtesy:~\cite{Chowdhury2022}.}
	    \end{figure}
	    
	    \item \bfnt{Screening under optical microscope for domains}: Polarized light is passed through a crystal, and the crystal is rotated about its axis. The transmitted light is observed through an analyzer. If, on rotating the analyzer, the crystals appears either uniformly dark or uniformly bright in all regions of the crystal, then the crystal has a single domain. Crystals with multiple domains would appear both dark and bright, or show multiple colours at certain angles between the polarizer and analyzer. This is demonstrated in figure~\ref{fig:crystal_optical_micro}.
	    
	\end{itemize}
	
		We cannot know whether a crystal is good or bad until we mount it on the diffractometer and put it in the path of X-rays. Therefore, when a crystal passes the above minimum selection criteria, we mount it on the goniometer head and shine X-rays on it. The axis about which the crystal is mounted is known as the $\phi$ axis. We rotate crystal about the $\phi$ axis through a full $2\pi$ rotation while keeping the X-ray on for 1-2 minutes depending on the size of the crystal, and we keep recording the diffraction pattern. The diffraction pattern thus recorded is known as the \bfnt{rotation photograph} of the single crystal.
	
	\begin{figure*}
	\centering
	\includegraphics[scale=0.3]{rotation_photograph_mod.png}
	\caption{\label{fig:rotation_photo}Rotation photograph of a single crystal of $\mathrm{NaCl}$. Each spot has another spot on the other side related by a centre of inversion. Some examples are shown by different coloured circles; two spots encircled by the same colour are related to each other. The dark shadow is that of the beam stop, which prevents the direct X-ray beam from falling on the detector. Image courtesy:~\cite{Chowdhury2022}.}
	\end{figure*}
	
	Figure~\ref{fig:rotation_photo} shows the rotation photograph of a single crystal. This photograph is always centrosymmetric irrespective of the type of crystal geometry. Each spot on the rotation photograph has another corresponding spot that is related to it by a centre of inversion. If instead of these spots, the rotation photograph is composed of concentric circles, we conclude that the crystal is not a single crystal, but a polycrystal, and is not suitable for XRD.
		
		\subsection{Mounting a crystal on the diffractometer}

	\begin{figure*}[t]
		\includegraphics[scale=0.5]{goniometer1.png}
		\caption{\label{fig:goniometer}The goniometer head, mounting loop with the mounting base.}
	\end{figure*}

	In fig.~\ref{fig:goniometer}, a goniometer head is shown. This head is mounted on the goniometer of the diffractometer. The top of the goniometer head is a magnetic base which allows the mounting loops to be held in place. To align the crystal in the X-ray beam, the goniometer head has three screws that allow movement in the three Cartesian directions. These screws and their respective directions have been demarcated in the figure.

		The tip of the mounting metal pin has a small polymer loop. The crystal is mounted on this loop using some thick oil, which keeps the crystal in place by its surface tension. These loops are available in various diameters, generally in the range $0.05-0.5~\si{mm}.$

		Nylon loops are also available, in which the loop is made by a nylon thread, which is then twisted several times and then glued to a pin. The pin is attached to a brush, which can be directly mounted on the goniometer head and placed in the X-ray beam.
	
		\subsection{The diffractometer}

\begin{figure}
	\centering
	\includegraphics[width=\textwidth]{sc_diffractometer_lateral.png}
	\caption{\label{diffractometer_lateral}Lateral view of a SCXRD diffractometer. See text for details.}
\end{figure}

Figure~\ref{diffractometer_lateral} is a lateral schematic of the diffractometer. The X-rays are generated within the tube shield and are emitted through the transparent Be window. When we physically work with the diffractometer, for example, to mount the crystal, we do not want the X-rays in the diffractometer chamber. Shutter \# 1 on the tube shield is used to stop the X-rays from coming out of the tube shield. The beam pipe contains the beam optics, which consists of the $K_\beta$ filter, collimation optics, as well as another shutter. This shutter controls the exposure time.

The intensity of the generated X-rays is very high, and all of the intensity is not diffracted by the crystal. A portion of the X-ray beam transmits through the crystal and directly hits the detector if it is at $2\theta = 0.$ This high intensity beam can damage the detector. The beam stop prevents the direct beam from falling on the detector. For a typical beam of spot size $\SI{0.5}{mm}$ diameter, the beam stop is $\sim \SI{2}{mm}.$ In Fig.~\ref{fig:rotation_photo}, the dark portion is the shadow of this beam stop.

The distance between the detector and the crystal, denoted by $d,$ is generally kept $4-10~\si{cm},$ and the detector can move back upto $\SI{18}{cm}.$ The intensity of the diffracted beam falls off $\propto \dfrac{1}{d^6}.$ Hence, if $d$ is large, exposure time will increase. $d$ depends on the lattice parameters of the crystal:%
%	
	\begin{subequations}
		\begin{align}
		d \sim \begin{cases}
		\SI{4}{cm} & \text{for } a, b, c < 12-13~\si{\angstrom};\\
		5-6~\si{cm} & \text{for } a, b, c < 12-30~\si{\angstrom};\\
		7-8~\si{cm} & \text{for } a, b, c > \SI{30}{\angstrom}.\\
		\end{cases}
		\end{align}
	\end{subequations}
	
\begin{figure}
	\centering
	\includegraphics[width=\textwidth]{sc_diffractometer_omega.png}
	\caption{\label{diffractometer_omega}Top view of a SCXRD diffractometer. The $\omega$ axis is the rotation axis of the diffractometer base. $2\theta$ is the angle of rotation of the detector.}
\end{figure}
	
\begin{figure}
	\centering
	\includegraphics[width=\textwidth]{sc_diffractometer_phi_chi.png}
	\caption{\label{diffractometer_phi_chi}The $\phi$ axis is the rotation axis of the crystal about the mounting loop. If the angle that the goniometer makes with the diffractometer base is fixed, then the angle is termed as the $\chi$ axis, with $\chi = \SI{54.7}{\degree}.$ If this angle can be changed, then the same axis is known as the $\kappa$ axis.}
\end{figure}

\begin{figure}
	\centering
	\includegraphics[width=\textwidth]{all_axes.png}
	\caption{\label{diffractometer_all_axes}All the axes of a single crystal diffractometer.}
\end{figure}
 
A diffractometer can have four axes of rotation:%
%	
	\begin{enumerate}%
%	
	    \item $\phi$ axis,
	    
	    \item $\omega$ axis,
	    
	    \item $\chi$ or $\kappa$ axis, and
	    
	    \item $2\theta$ axis.
	    
	\end{enumerate}

These axis are shown in figures~\ref{diffractometer_omega}, \ref{diffractometer_phi_chi} and \ref{diffractometer_all_axes}.\\

Based on this, SCXRD diffractometers can be classified into:%
%	
	\begin{itemize}%
%	
	    \item \ul{2-circle diffractometer}: $\phi$ and $\omega$ can be varied, but $\chi$ and $2\theta$ are fixed. $2\theta$ is fixed at $\SI{30}{\degree}.$ But the coverage area of the detector is such that if it sits at $\SI{30}{\degree},$ it can actually reach $0-60\si{\degree}.$ For $\mathrm{Mo}~K_\alpha$ radiation, the IUCr prescribed minimum $2\theta = \SI{50}{\degree}.$ So, a 2-circle diffractometer works fine for $\mathrm{Mo}~K_\alpha$ radiation. We can, however, not go to higher angles, so this diffractometer is used for routine analysis of crystals.
	    
	    \item \ul{3-circle diffractometer}: $\phi$, $\omega$ and $2\theta$ can be varied. Allows data recording upto high angles.
	    
	    \item \ul{4-circle diffractometer}: $\phi$, $\omega$, $2\theta$ and $\kappa$ can be changed. Reduces data collection time because a large number of reflections can be brought to the periphery of the Ewald sphere.
	    
	\end{itemize}
	
\begin{table}
	\newcounter{sl_no}
	\centering
	\caption{\label{tab:3_circ_diff_angles}Standard measurement angles and exposure time for a 3-circle diffractometer. $\Delta \omega$ is the step size or width in $\omega.$ The full set represents a complete sphere of data; the first two sets are recorded with 200 frames of the third set correspond to a hemisphere of data.}
	\begin{tabular}{|c|C|C|C|C|C|C|C|}
		
		\hline
		
		Sl. No. & 2\theta & \omega & \phi & \multicolumn{1}{c|}{\makecell{$\chi$\\(Fixed)}} & \Delta \omega & \multicolumn{1}{c|}{\makecell{\text{No. of frames}}} & \multicolumn{1}{c|}{\makecell{\text{Exposure time}\\$t$ (in s)}}\\
		
		\hhline{|=|=|=|=|=|=|=|=|}
		
		\stepcounter{sl_no}\arabic{sl_no} & \SI{-30}{\degree} & \SI{-30}{\degree} & 0 & \SI{54.74}{\degree} & \SI{1}{\degree}/\SI{0.5}{\degree}/\SI{0.3}{\degree} & 180/360/600 & 5/10/15\\
		
		\hline
		
		\stepcounter{sl_no}\arabic{sl_no} & \SI{-30}{\degree} & \SI{-30}{\degree} & \SI{90}{\degree} & \SI{54.74}{\degree} & \SI{0.3}{\degree} & 600 & 10\\
		
		\hline
		
		\stepcounter{sl_no}\arabic{sl_no} & \SI{-30}{\degree} & \SI{-30}{\degree} & \SI{180}{\degree} & \SI{54.74}{\degree} & \SI{0.3}{\degree} & 600 & 10\\
		
		\hline
		
		\stepcounter{sl_no}\arabic{sl_no} & \SI{-30}{\degree} & \SI{-30}{\degree} & \SI{270}{\degree} & \SI{54.74}{\degree} & \SI{0.3}{\degree} & 600 & 10\\
		
		\hline
	
	\end{tabular}
\end{table}

	
Data collection strategies vary based on the type of diffractometer and the crystal being studied. Table~\ref{tab:3_circ_diff_angles} lists the standard angles which allow a full sphere of data to be collected using a 3-circle diffractometer. The exposure time is the time period for which the X-ray is allowed to fall on the crystal before its orientation is changed. The crystal is not kept stationary, instead, it is slowly rotated about the $\omega$ axis in steps of $\Delta\omega.$ A large value of $\Delta\omega$ implies that we are slicing the reciprocal lattice into wider slices. If we take $\Delta\omega = \SI{1}{\degree},$ it is recommended that we use a larger exposure time, so that the exposure time per degree remains fixed. Smaller values improve the quality of data. If $\chi$ (i.e. $\kappa$) can be varied, we can get the data faster.

Diffractometers have pre-fixed strategies that are used when an unknown crystal is loaded for the first time. This strategy cannot be edited, but may be extended in some cases. Using this strategy, 20-30 frames are recorded, from which the diffractometer informs us about the lattice parameters $a, b, c, \alpha, \beta$ and $\gamma,$ and the type of lattice (P, I, F, C). From the lattice parameters, we can get the volume of the unit cell, $V.$ $V / Z$ is the volume of the asymmetric unit, i.e. the smallest unit cell that can be used to represent the lattice. This asymmetric unit must contain the total number of atoms.

Let $n$ be the number of non-Hydrogen atoms present in the molecule of interest.%
%
\begin{equation}
\therefore \text{Average atomic volume} = \dfrac{V}{Zn}.
\end{equation}

This average atomic volume should range between $16-20~\si{\angstrom}.$ If the measured value does not fall within this range, we have to check if the asymmetric unit has solvents, and take into account the number of Hydrogen atoms in the solvent. If the value still does not match, we have to discard the crystal and start over again.
		
		\subsection{Detectors}

Three types of detectors are available: charge coupled devices (CCD), CMOS and Direct Photon Counting Detectors.

CCD and CMOS contain photo sites or pixels. When light falls on a pixel, charge accumulates, which gives rise to a potential difference. This potential difference is amplified and then recorded.

In CCD, the pixels are read column-wise one after the other. Hence, it takes time to read all the pixels. CMOS technology, on the other hand, can simultaneously read data from all pixels. Hence, CMOS is preferred over CCD.

Temperature plays a significant role in the quality of data collected. If we carry out the XRD at room temperature of $\sim \SI{300}{K},$ the latice ions will vibrate in all the three directions due to thermal energy. Thus, the electron density associated with each of those atoms will be largely diffused over the plane of the atom. Due to this, the electron planar density is decreased and the diffraction will be weak. Therefore, we generally carry out XRD at $\SI{100}{K}$ using liquid nitrogen cryogen, or at $6-10~\si{K}$ using liquid He.
		
		\subsection{How much data do we have to collect?}

Understanding how much data we have to collect for different crystal systems was extremely essential even, say, 20 years back. In those days, the detector in the diffractometer used to be a point detector. Such a detector can record only one reflection at a time, and it would take quite a long time to center a reflection at the maxima and then record the intensity. 

When we mount a crystal on the diffractometer and rotate it about the $\phi$ axis, diffraction would be visible from all directions. The detector, however, being a point detector, could only measure in a particular plane. The procedure used was as follows: First, the detector would be moved to the location of a particular reflection. Next, the detector would be moved in very small steps to locate the maxima of the diffracted light. Thereafter, the crystal would be very slowly rotated about the different axes of the diffractometer to maximize the intensity once again. Thus, for recording a particular reflection, it would take 5-7 minutes. If the intensity is weak at that point, it would take at least 15 minutes to get sufficient data. So, without Friedel's Law, one would have to spend more than a week to record all reflections.

Suppose, for a tricilinic system, 10,000 reflections are possible considering all $(h,k,\ell)$ and $(\bar{h}, \bar{k}, \bar{\ell}).$ It would take around 6-8 days for this data to be collected. Thanks to Friedel's Law, we know that we have to collect only half of the data rather than the full sphere. If we know whether the crystal is monoclinic, triclinic or orthorhombic, data collection time reduces further.

Even with the advent of modern area detectors, Friedel's Law is still equally useful.
		
		\subsection[Data collection and indexing (an overview)]{Experimentation, data collection and refinement, and indexing --- \\a preliminary overview}

At the time of writing, there are two major diffractometer manufacturers in the world:%
%	
	\begin{itemize}%
%	
	    \item Bruker Axs, and
	    
	    \item Rigaku Oxford Diffraction.
	    
	\end{itemize}
	
Both of these manufacturers provide single crystal as well as powder X-ray diffractometers. They have their own software for controlling the diffractometers during the experiment, as well as for analysis, indexing and structure solution of the data recorded by it.

Data refinement and structure solution are done through pre-programmed software nowadays, and it would really not be possible to go into the full details of each and every software and how data refinement and structure indexing are done. We are going to scratch the surface and very briefly mention an approximate procedure that is followed; for details, the reader is advised to go through any books dealing in X-ray crystallography from the experimental aspect, like \cite{Cullity2014}. The procedure that we mention is primarily taken from \cite{Chowdhury2022} and based on a 4-circle single crystal diffractometer from Rigaku Oxford Diffraction, which also has a cryosystem from Oxford Cryosystems for maintaining the temperature of the crystal at cryogenic scales, if necessary, and a $CO_2$ laser for carrying out in situ crystallization.

All diffractometers come with an optical microscope and a video camera through which the crystal can be viewed in a software. The first step after mounting the crystal is \bfnt{centring the crystal}. This is highly essential and should be done precisely, because during the experiment, the crystal would be rotated through various angles, and it is necessary to make sure that there is no position in which the crystal does not receive the X-ray beam. The crystal is rotated through the centre, left, right and top orientations (w.r.t. the microscope) using the control software, and the screws on the goniometer head are used to move the crystal, which is again observed on the software to make sure that the crystal is centred.

Next step is to capture a \bfnt{rotation photograph}. For this, we use a preset zero position, select $\phi$ to be rotated through $2\pi,$ and record the data.

Based on the rotation photograph generated by the software, we may have to change the exposure time if spots are overlapped. To do this, we take a still photograph of a certain exposure time and check if the spots are well separated.

After finding a good exposure time, we again execute a preset scanning strategy to collect three sets with 20-30 frames each for \bfnt{unit cell determination}. The strategies are calculated such that at least three regions of the reciprocal lattice are scanned for precise measurement of the unit cell. The indexed percentage is important because anything below 60\% indexed is not acceptable, while a value above 80\% is considered to be good. Once the recording is over, we choose an appropriate value of mean~$I/\sigma$ (signal to noise ratio) to include lower intensity spots, as needed, and ``harvest'' the spots. We use different methods like difference vectors, FFT and least squares on the chosen reflections, and proceed to determine the unit cell parameters.

The software gives us the unit cell parameters calculated using the chosen methods, along with a score. If the data is good, the three methods should give nearly the same values, though the determined unit cell might be different. We choose the algorithm that has the highest score, or the unit cell with the highest symmetry, and thereafter refine the data. We can decrease the tolerance if we want to include more number of reflections. At this stage, different histograms can be viewed. A single tall peak in the rotation angle histogram tells us that our crystal was properly centred, while a Gaussian-like distribution indicates that the centring was not perfect. Histograms can be viewed for $h$, $k$ and $\ell$ values, through which we can learn whether our crystal was truly a good single crystal. Thereafter, the software tells us the possible Bravais lattices with figures of merit for each.

Next, we refine the data once again. At this stage, we can \bfnt{view the reflections on the reciprocal lattice} using the ``Reciprocal lattice viewer". For a good single crystal, the spots should appear perfectly aligned with each other in a grid when viewed along the three reciprocal lattice axes. We can select and remove any spots that appear to be mis-aligned at this stage.

Now we are ready to collect the main data. We go to the ``Data Collection Strategy'' module, and choose the resolution that we want, and the symmetry. We can just choose between `chiral',  `centrosymmetry' and different Laue groups. The software also asks about the detector distance that should be used.

Based on our input, the software will compute the best data collection strategy, and present us with a number of runs. Each run collects data upto a certain completeness level, which is shown to us. If we set the frame width ($\Delta \omega$) and the exposure time, the software will also tell us how much time is needed for each of the runs. Each run sweeps a certain diffractometer axis with other axes fixed.

Now, we go to the ``Experiment'' module and ``Append'' the strategy. We must validate the strategies to make sure that the diffractometer collision limits are not exceeded. Once the strategies are validated, we can start the data collection.

The recorded data is stored in files such that each file corresponds to one frame. Two consecutive frames have a difference of $\Delta \omega.$ We open the \texttt{.par} file in a data management software like \texttt{Crys Alis Pro}. First, we re-calculate the unit cell parameters. Next, we view the spots in the reciprocal lattice viewer, and remove the sets of misaligned spots. In most cases, the software will already group the misaligned spots, and we just have to remove them by de-selecting the group.

Using this same software, we can perform data reduction using various strategies, and at the end of this process, the software will ask us to choose the lattice type and the space group from a set of options. We are also required to enter an approximate formula of the crystal molecule.

With all this data, the software completes the computation and generates a \texttt{Mercury} file (\texttt{Mercury} is a software from the Cambridge Structural Database). This file, which can also be opened on a text editor, contains all the required information about the crystal system.

For structure solution, we use the \texttt{Olex 2} software. This is outside the scope of this article.

	
	\section{Powder X-ray Diffraction}
	
		\subsection{Instrumentation}

	PXRD diffractometers can be broadly classified into two types:%
%	
	\begin{enumerate}%
%	
	    \item the reflection geometry, and
	    
	    \item the transmission or Debye-Scherrer geometry.
	    
	\end{enumerate}
	
	\begin{figure}
		\centering
		\includegraphics[scale=0.15]{bragg_bretano_diff.png}
		\caption{\label{fig:bragg_bretano_diff}Schematic of a reflection-type Bragg-Bretano diffractometer.}
	\end{figure}
	
	Figure~\ref{fig:bragg_bretano_diff} shows the schematic of a reflection geometry-based powder X-ray diffractometer. The sample holder is rotated in the presence of the beam to remove any special orientation effect. In PXRD, we do not use a circularly focussed beam, instead, we use a line beam, with the line falling on the sample. There are two geometries based on how the line beam is created:%
%	
	\begin{enumerate}%
%	
	    \item The Bragg-Brentano (BB) geometry, which uses a diverging beam.
	    
	    \item The parallel beam (PB) geometry, which uses a parallel beam of X-rays.
	    
	\end{enumerate}
	
	\begin{figure}
		\centering
		\includegraphics[width=\textwidth]{cross_beam_optics.png}
		\caption{\label{fig:cross_beam_optics}Bragg-Bretano (BB) and parallel beam (PB) geometry. Cross Beam Optics is a patented technology by Rigaku Oxford Diffractometers. The BB and PB geometries are simultaneously mounted, aligned and selectable by the means of a removable slit. Design taken from \cite{Chowdhury2022}.}
	\end{figure}
	
	This Cross Beam Optics is shown in figure~\ref{fig:cross_beam_optics}. The PB geometry is particularly useful when we have a small amount of sample, and we want to focus the X-rays on the sample only and want to avoid the diffraction of the sample holder plate.
	
	 The Bragg-Brentano diffractometers can further be of two types, as listed in table~\ref{tab:bragg_bretano}
	 . In the $\theta : 2\theta$ type, the X-ray source is fixed, while the specimen holder and the detector are both rotated. The detector moves by an angle $2\theta$ when the sample holder moves by $\theta.$ In the $\theta : \theta$ geometry, the specimen is kept fixed, while the tube and the detector are moved together in opposite directions (i.e. both towards or away from each other).
	 
	\begin{table}
	
		\centering
		
		\caption{\label{tab:bragg_bretano}Types of Bragg-Bretano diffractometers.}
		\begin{tabular}{|c|c|c|c|}
		
			\hline
			
			Type & X-ray tube & Specimen holder & Detector \\
			
			\hhline{|=|=|=|=|}
			
			$\theta : 2\theta$ & Fixed & Varies as $\theta$ & Varies as $2\theta$ \\
			
			\hline
			
			$\theta : \theta$ & Varies as $\theta$ & Fixed & Varies as $\theta$ \\
			
			\hline
		
		\end{tabular}
	\end{table}
	
	The Debye-Scherrer diffractometers are transmission-type diffractometers in which the X-rays pass through the sample and are recorded by the detector on the other side. These are used when the sample does not absorb X-rays, and are particularly useful if the sample is loaded in a capillary tube.
	
\subsection{Data collection}

	Collecting high quality data is the primary requirement in PXRD experiments. We need to keep a few points in mind in these experiments:%
%		
		\begin{enumerate}%
%		
		    \item \bfnt{Sample quality}: The sample must be free-flowing powder, must not be moisture sensitive, and must ideally contain no moisture. Presence of moisture will change the background in the recorded data.
		    
		    \item \bfnt{Range of $2\theta$}: This will not be known \textit{a priori}. Initially, we collect data upto an approximate range, and based on the peaks, we decrease the step size and focus on a certain range. For organic and organometallic samples, we generally record in the range $3-50~\si{\degree}$ in $2\theta,$ while for inorganic samples, we often go up to around $\SI{100}{\degree}.$
		    
		    \item \bfnt{Scan speed}: Scan speed can vary from as high as $\SI{20}{\degree}$ in $\si{2\theta \per min}$ to as low as $\SI{0.2}{\degree}$ in $\si{2\theta \per min}.$
		    
		    \item \bfnt{Step size}: Step sizes can vary between $\SI{0.1}{\degree}$ in $2\theta$ to as low as $\SI{0.001}{\degree}$ in $2\theta.$ We must ensure that the scan speed and step size combination is such that sufficient time is allowed to resolve each and every peak.
		    
		    \item \bfnt{Sample rotation}: Due to spreading process of the powdered sample on the sample plate, it may happen that one side of the sample is generating unusually higher peaks compared to the other. This is called the preferred orientation effect. It is necessary to rotate the sample to get rid of this. Normally, a rotation speed of $100-120~\si{rpm}$ speed is used.
		    
		    \item \bfnt{Monochromator}: Generally, PXRD diffractometers are not equipped with a \\monochromator, but we can use one. However, the intensity of the peaks reduces drastically in that case.
		    
		\end{enumerate}
		
	Overlapping of peaks is one of the very common issues in PXRD, and it is very difficult to separate such overlapped peaks. Indexing is a much harder work, and is often done using the Inorganic Crystal Structure Database (ICSD) and the Organic Powder Structural Database, maintained by the International Centre for Diffraction Data (ICDD). Using this database, impurities can also be identified in many cases. We shall not go into the details of indexing using software for PXRD.
	
		
	\section{Use of XRD in characterization of nanomaterials}

Powder XRD is commonly used in characterization of nanomaterials. Analysis of a powder sample by PXRD provides important information like phase identification, sample purity, crystallite size, and sometimes, even the morphology.~\cite{Holder2019}.

The primary issue with using XRD in nanoscale systems is the peak broadening with the decrease in the crystallite size. As long as adjacent peaks are not overlapping, the Scherrer equation can be used to find the crystallite domain size. Below $\SI{10}{nm},$ however, peak broadening is so significant that signal intensity is low, and peaks often overlap. For domain sizes below $\SI{5}{nm},$ it is difficult to analyze the data, both because of broad peaks and low signal-to-noise ratio.

Size-dependent XRD peak broadening has important implications for nanomaterial characterization. For instance, if TEM analysis shows spherical particles having an average diameter of $\SI{10}{nm},$ but the XRD pattern has sharp peaks that are more consistent with particles having much larger crystalline domain sizes, then the majority of the bulk sample is not composed of $\SI{10}{nm},$ particles; it is more likely that the microscopically observed $\SI{10}{nm},$ particles represent only a minority subpopulation.

Not all nanoparticles are spherical. In case of non-spherical nanoparticles, there is a chance that upon drying, they will orient in non-random directions. As noted in~\cite{Holder2019}, a sample of cube-shaped particles dried or precipitated from solution will tend to orient with their flat faces parallel to the drying surface. It is much less likely that nanocubes would dry with their corners or edges touching the drying surface, and therefore, the powder of nanocubes will be preferentially oriented in the crystallographic direction corresponding to the faces. Similarly, one-dimensional nanowires will tend to orient flat on a substrate upon drying. Other particle shapes, such as octahedra or tetrahedra, may have different ways of orienting. The majority of the sample may exhibit preferred orientation, or only a fraction of it may, depending on the quality and size of the various particle shapes. In addition, the method in which the sample was dried to form a powder and/or how the XRD sample was prepared can influence the preferred orientation of the sample. We have already discussed the preferred orientation effect in PXRD. This can then be utilized to gain information about the structure of the crystallized nanoparticles.

Another application of PXRD is \bfnt{phase identification}, which is often accomplished by comparing an experimental XRD pattern with a reference pattern that is either simulated or obtained from a database. In such cases, an unambiguous and complete match between the experimental and reference patterns is needed. Arbitrary peaks predicted by a reference pattern cannot be missing in the experimental XRD data without justification. All peaks in the reference pattern, which includes both of their diffraction angles and intensities, should be accounted for in the experimental pattern unless there is a clear and justified rationale for why certain peaks may be missing or have different intensities, such as preferred orientation, as discussed above. To accomplish this comparison, experimental XRD patterns having sufficient signal-to-noise ratios are needed so that low-intensity peaks can be observed.

Phase identification by XRD for some systems, especially nanoscale materials, can be particularly challenging because of nearly indistinguishable diffraction patterns. For example, Au and Ag are both face-centered cubic metals that have sufficiently similar lattice constants that Au and Ag nanoparticles (which have broadened peaks) cannot be differentiated by XRD.

PXRD patterns for crystalline and amorphous samples are highly dissimilar. This can be used to detect the presence of amorphous impurities in the sample. However, samples that produce XRD patterns having low signal-to-noise ratios, including poorly crystalline materials and nanoscale materials having significantly broadened peaks, can contain large amounts of components that do not produce XRD peaks that rise significantly above the background noise. Low-intensity peaks, which may correspond to impurities, can also be difficult to observe. The presence of asymmetric peaks may be due to stacking faults and other defects or a distribution of compositions in compounds that could be present as alloys or solid solutions.

For bulk crystalline samples, lattice constants can be calculated upto four decimal places. However, for nanoscale materials, peak broadening and low signal-to-noise ratio does not allow the values to be so precise.
	
	\medskip
	
	%\bibliographystyle{plain}
	%\bibliography{bibliography}
	\printbibliography
		

\end{document}