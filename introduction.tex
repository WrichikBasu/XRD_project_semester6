\section{Introduction}

X-ray crystallography is the experimental science of determining the structure of crystals using X-rays. The discovery of diffraction of X-rays by crystals by Max von Laue, and subsequent work by the Braggs, Ewald and Scherrer, opened a vast realm in science. A large number of substances in the nature, including biological molecules, can crystallize, and hence can be studied using this technique. Not only can we determine the crystal structure of the sample, we can also find the structure of the material under study. In this writing, we start by first talking about the generation and sources of X-rays. Next, we briefly dive into the theory of X-ray diffraction before entering the experimental realm. Thereafter, we discuss which type of X-rays are used and the types of XRD experiments. Then comes the two major types of XRD experiments. We conclude with the application of XRD in nanophysics. We hope that you will enjoy this journey. Without further ado, let us begin.