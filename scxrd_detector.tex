\subsection{Detectors}

Three types of detectors are available: charge coupled devices (CCD), CMOS and Direct Photon Counting Detectors.

CCD and CMOS contain photo sites or pixels. When light falls on a pixel, charge accumulates, which gives rise to a potential difference. This potential difference is amplified and then recorded.

In CCD, the pixels are read column-wise one after the other. Hence, it takes time to read all the pixels. CMOS technology, on the other hand, can simultaneously read data from all pixels. Hence, CMOS is preferred over CCD.

Temperature plays a significant role in the quality of data collected. If we carry out the XRD at room temperature of $\sim \SI{300}{K},$ the latice ions will vibrate in all the three directions due to thermal energy. Thus, the electron density associated with each of those atoms will be largely diffused over the plane of the atom. Due to this, the electron planar density is decreased and the diffraction will be weak. Therefore, we generally carry out XRD at $\SI{100}{K}$ using liquid nitrogen cryogen, or at $6-10~\si{K}$ using liquid He.